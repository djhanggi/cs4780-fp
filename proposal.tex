%%%%%%%%%%%%%%%%%%%%%%%%%%%%%%%%%%%%%%%%%%%%%%%%%%%%%%%%%%%%%%%%%%
%%%%%%%% ICML 2012 EXAMPLE LATEX SUBMISSION FILE %%%%%%%%%%%%%%%%%
%%%%%%%%%%%%%%%%%%%%%%%%%%%%%%%%%%%%%%%%%%%%%%%%%%%%%%%%%%%%%%%%%%

% Use the following line _only_ if you're still using LaTeX 2.09.
%\documentstyle[icml2012,epsf,natbib]{article}
% If you rely on Latex2e packages, like most moden people use this:
\documentclass{article}

\title{CS4780_Final_Project_Proposal}

% For figures
\usepackage{graphicx} % more modern
%\usepackage{epsfig} % less modern
\usepackage{subfigure} 

% For citations
\usepackage{natbib}


% For algorithms
\usepackage{algorithm}
\usepackage{algorithmic}

% As of 2011, we use the hyperref package to produce hyperlinks in the
% resulting PDF.  If this breaks your system, please commend out the
% following usepackage line and replace \usepackage{icml2012} with
% \usepackage[nohyperref]{icml2012} above.
\usepackage{hyperref}
\hypersetup{
    colorlinks=true,
    linkcolor=magenta,
    filecolor=magenta,      
    urlcolor=magenta,
}
% Packages hyperref and algorithmic misbehave sometimes.  We can fix
% this with the following command.
\newcommand{\theHalgorithm}{\arabic{algorithm}}

% Employ the following version of the ``usepackage'' statement for
% submitting the draft version of the paper for review.  This will set
% the note in the first column to ``Under review.  Do not distribute.''
\usepackage{icml2012} 
% Employ this version of the ``usepackage'' statement after the paper has
% been accepted, when creating the final version.  This will set the
% note in the first column to ``Appearing in''
% \usepackage[accepted]{icml2012}

% \setlength\parindent{18pt}


% The \icmltitle you define below is probably too long as a header.
% Therefore, a short form for the running title is supplied here:
\icmltitlerunning{Submission and Formatting Instructions for ICML 2012}

\begin{document} 

\twocolumn[
\icmltitle{CS 4780 Final Project Proposal: What Makes a Song "Beatlesque"?}

% You may provide any keywords that you 
% find helpful for describing your paper; these are used to populate 
% the "keywords" metadata in the PDF but will not be shown in the document
\icmlkeywords{machine learning, influencers, The Beatles, music}

\vskip 0.1in
]

\section{Size of Team}
There are four (4) members in our team. 

\section{Motivation}
The Beatles are considered one of the most influential bands of all time. We think it will be an interesting and challenging task to formalize how that influence is manifested in the metadata of the work of other artists. 
\section{Problem Statement}
%The end goal for our project is to classify whether or not an artist was influenced by The Beatles, given a set of songs by that artist.\par
It is difficult to formalize what being "influenced by The Beatles" means, so our main learning task is to determine the relationship between a song's metadata and the extent that the song (and by extension, its artist) is influenced by the work of The Beatles.\par
More concisely, we want to determine how similarity in song metadata translates to musical influence.\par
%We will determine this influence based on metadata from songs in the Million Song Dataset (described in a later section).\par

%the genre and decade of a given song at at least the same or better accuracy than a typical person. That is because our algorithm will be able to incorporate rich metadata of songs, such as "danceability" or "energy levels", "beats per measure", etc. that we as humans don't have access to, to give them powerful tools to help with this classification task.  
\section{General Approach}
\subsection{Background Knowledge}
This will be a supervised learning problem, so we need to acquire background information on different ways musical influence manifests itself in a song.

The Beatles have a large corpus of music (over 240 songs), so we will study the 25 most popular songs by The Beatles in our training, testing, and analysis.

\subsection{Data Parsing and Clean Up} 
The \href{http://labrosa.ee.columbia.edu/millionsong/}{Million Song Dataset} includes attributes unnecessary for our goals, so we will only focus on certain features such as artist familiarity, tags describing the artist, energy, year, and similar artists. We are open to using more attributes and will learn more important ones in our main learning exercise.

\subsection{Learning Task Setup} 
For our first round of training, we will use 200 sample songs - 50 of which we have already determined as "influenced" by the Beatles, in order to develop better intuition for the nature of The Beatles' influence and begin understanding how to compute weights in the features before officially training and testing on our remaining data. An additional 1,000 songs will be used for second-round training and validation. 1,000 songs will be used for testing.  

\subsection{Training}
We will test various similarity measures on small subsets of our data to learn what features are most important in determining The Beatles' influence on a song. \par
Once we find a strong similarity measure, we will implement the K Nearest Neighbor algorithm to provide a classification of an artist given the metadata and similarity of songs by that artist to The Beatles.\par

% \subsection{Evaluation}
% Given the training results, we will research a large subset of our results to determine the extent of The Beatles' influence to better articulate our results.

\section{Resources}
Dataset: \href{http://labrosa.ee.columbia.edu/millionsong/}{Million Song Dataset}, Columbia University.\\
Language: Matlab, useful data visualization packages. \\
Version Control: git, private repository on GitHub.

\section{Schedule}
\begin{table}[H]
\label{sample-table}
\vskip 0.1in
\begin{center}
\begin{small}

\begin{tabular}{ll}
\hline
\abovespace\belowspace
Date & Milestone \\
\hline
\abovespace\belowspace
\textbf{10/21} & \textbf{Project Proposal Due}\\
\belowspace
10/26 & Finish Background Research\\
\belowspace
11/7 & Data Parsing*\\
\belowspace
11/9 & Influence Linked to Song Attributes*\\
\belowspace
\textbf{11/11} & \textbf{Progress Report Due}\\
\belowspace
11/15 & Testing of Various Similarity Measures*\\
\belowspace
11/25 & Implementation of KNN \& Testing\\
\belowspace
11/28 & Evaluation of Findings\\
\belowspace
12/02 & Finish Final Report**\\
\belowspace
12/02 & Finish Poster**\\
\belowspace
\textbf{12/04} & \textbf{Poster Presentation}\\
\belowspace
\textbf{12/16} & \textbf{Final Project Report Due}\\
\hline
\end{tabular}
\end{small}
\end{center}
\vskip -0.1in
\end{table}
*, ** Tasks occurring concurrently\\
\textbf{Bold face text} Hard deadline from course instructors

\end{document} 


% This document was modified from the file originally made available by
% Pat Langley and Andrea Danyluk for ICML-2K. This version was
% created by Lise Getoor and Tobias Scheffer, it was slightly modified  
% from the 2010 version by Thorsten Joachims & Johannes Fuernkranz, 
% slightly modified from the 2009 version by Kiri Wagstaff and 
% Sam Roweis's 2008 version, which is slightly modified from 
% Prasad Tadepalli's 2007 version which is a lightly 
% changed version of the previous year's version by Andrew Moore, 
% which was in turn edited from those of Kristian Kersting and 
% Codrina Lauth. Alex Smola contributed to the algorithmic style files.